% Preamble
\documentclass[xcolor=dvipsnames]{beamer}
\usetheme{madrid}

% Packages
\usepackage[english,ngerman]{babel}
\usepackage[utf8]{inputenc}
\usepackage{amsmath}
\usepackage{graphicx}
\usepackage{amssymb}

\definecolor{hBlue}{RGB}{55,118,165}
\usecolortheme[named=hBlue]{structure}

\titlegraphic{\includegraphics[width=4cm]{../images/logo.png}}
\title{Gesundheit \& Ernährung}
\subtitle{Das Verdauungsrohr}
\author{Adrian Helberg}
\date{19.05.2021}

% Document
\begin{document}

    \maketitle

    \frame{\frametitle{Agenda}\tableofcontents}

    \section{Theorie}
    {
    \setbeamercolor{normal text}{fg=hBlue}\usebeamercolor*{normal text}
    \begin{frame}
        \begin{center}
            \Huge Theorie
        \end{center}
    \end{frame}
    }

    \subsection{Mund}
    \begin{frame}[allowframebreaks]
        \frametitle{Der Mund}

        \begin{block}{Aufgaben}
            \begin{itemize}
                \setlength\itemsep{1em}
                \item Berührungspunkt Nahrung und Verdauungssystem
                \item Oberflächenvergrößerung der Nahrung
                \item Mischen von Speichel und Nahrung
                \item Schlucken, Reinigen, Betäuben, Untersuchen (Immungewebe)
            \end{itemize}
        \end{block}

        \framebreak

        \begin{block}{Speichel}
            \begin{itemize}
                \setlength\itemsep{1em}
                \item 0,7-1 l / Tag (nachts weniger, als tagsüber)
                \item Gefiltertes Blut mit Calcium, Hormonen, Abwehrstoffen, Enzymen und Opiorphin (entd. 2006)
                \item Schutz der Zähne
                \item Opiorphin wirkt antidepressiv und betäubend
                \begin{itemize}
                    \item Korrelation mit "`Frustessen"' in aktueller Forschung
                    \item morgendliche Halsschmerzen, Linderung beim Lutschen, Kauen
                    \item 3-6x stärker als Morphium
                \end{itemize}
                \item Forschung: Korrelationen Zahnproblematik, Immunsystem und Übergewicht
            \end{itemize}
        \end{block}

        \framebreak

        \begin{block}{Mandeln}
            \begin{itemize}
                \setlength\itemsep{1em}
                \item Waldeyer Rachenring (Immungewebe)
                \item Mandelentfernung kann zu Übergewicht führen
                \item Autoimmune Erkrankungen mit Halsschmerzen deuten auf Mandelproblematik
                \begin{itemize}
                    \item Schuppenflechte
                    \item Rheuma
                \end{itemize}
                \item "`Versteckte"' Bakterien
            \end{itemize}
        \end{block}

    \end{frame}

    \subsection{Magen}
    \begin{frame}[allowframebreaks]
        \frametitle{Der Magen}

        \begin{block}{Aufbau}
            \begin{itemize}
                \setlength\itemsep{1em}
                \item Speiseröhre seitlich in den linksgekrümmten Magen
                \item Linke Brustwarze bis rechter Rippenbogen
                \item Darmbeschwerden werden häufig mit Darmbeschweren verwechselt
                \item Luft im Magen kann zu Herzstichen und Panikattacken führen
            \end{itemize}
        \end{block}

        \framebreak

        \begin{block}{Aufgaben}
            \begin{itemize}
                \setlength\itemsep{1em}
                \item Speichern, Homogenisieren, Zerkleinern
                \item Flüssigkeit läuft an der kurzen Seite direkt in den Darm, Nahrung gelang in den Magenbeutel
                \item Eiweiß stocken
            \end{itemize}
        \end{block}
    \end{frame}

    \subsection{Dünndarm}
    \begin{frame}
        \frametitle{Der Dünndarm}

        \begin{block}{Aufbau \& Aufgaben}
            \begin{itemize}
                \setlength\itemsep{1em}
                \item 3-6m Länge, ca. 7km Fläche
                \item Zerteilung der Nahrung auf den kleinsten gemeinsamen Nenner der körpereigenen Stoffe
                \item Aufnahme von Nährstoffen über Darmzotten und Lymphsystem
                \item Alle Darmzotten laufen zusammen und dann zur Leber zur Überprüfung der Nahrung
                \item Ausgabe von Sättigungsstoffen, um den Körper zu entspannen $\rightarrow$ mehr Energie für die Verdauung
            \end{itemize}
        \end{block}
    \end{frame}

    \subsection{Blinddarm}
    \begin{frame}[allowframebreaks]
        \frametitle{Der Blinddarm}

        \begin{block}{Aufbau}
            \begin{itemize}
                \setlength\itemsep{1em}
                \item Verbindung zwischen Dünn- und Dickdarm
                \item Endet im Wurmfortsatz
                \item Immungewebe
            \end{itemize}
        \end{block}

        \framebreak

        \begin{block}{Aufgaben}
            \begin{itemize}
                \setlength\itemsep{1em}
                \item Überprüfung der Nahrung auf Keime
                \item Stichprobe der Darmflora
                \item Neubesiedelung des Darms
                \item Wiederaufbau einer Darmflora
            \end{itemize}
        \end{block}
    \end{frame}

    \subsection{Dickdarm}
    \begin{frame}
        \frametitle{Der Dickdarm}

        \begin{block}{Aufgaben}
            \begin{itemize}
                \setlength\itemsep{1em}
                \item Symbiotisches System Mensch $\rightarrow$ Bakterien
                \item Bakterienkonzentration nimmt mit Voranschreiten des Darms zu
                \item Herstellung von Stoffen, wie Hormone, durch Bakterien
            \end{itemize}
        \end{block}
    \end{frame}

    \subsection{Makronährstoffe}
    \begin{frame}
        \frametitle{Makronährstoffe}

        \begin{center}
            \textit{Jedes Lebensmittel besteht aus den drei Makronährstoffen:\\ Kohlenhydrate, Proteine und Fette}
        \end{center}
    \end{frame}

    \begin{frame}[allowframebreaks]
        \frametitle{Kohlenhydrate}

        \underline{Kleinster Baustein: Zucker}
        \begin{block}{Fakten}
            \begin{itemize}
                \setlength\itemsep{1em}
                \item Ca. 2/3 der weltweiten Biomasse
                \item Kohlenhydrate enden meist auf "`-ose"'
                \item Keine Hydrate des Kohlenstoffs
                \item Energiedichte: 4 kcal/g
            \end{itemize}
        \end{block}

        \framebreak

        \begin{block}{Aufnahme}
            \begin{itemize}
                \setlength\itemsep{1em}
                \item Spalten in Zucker (Sowohl Nudeln und Haushaltszucker werden zum selben Zucker)
                \item Dauer der Spaltung $\rightarrow$ Glykämischer Index
                \item Aufnahme des Zuckers durch die Darmzotten im Dünndarm
                \item Transport über Blut in die Leber (Pfortader)
            \end{itemize}
        \end{block}
    \end{frame}

    \begin{frame}[allowframebreaks]
        \frametitle{Proteine}

        \underline{Kleinster Baustein: Aminosäuren}
        \begin{block}{Fakten}
            \begin{itemize}
                \setlength\itemsep{1em}
                \item Meist mehr als die Hälfte des Trockengewichts von Zellen
                \item Aufgaben wie Zellbewegung und Signalstoffe erkennen
                \item Überweiegend in Muskeln, Haut und Haaren
                \item Proteinsynthese in der Leber
                \item Energiedichte: 4 kcal/g
            \end{itemize}
        \end{block}

        \framebreak

        \begin{block}{Aufnahme}
            \begin{itemize}
                \setlength\itemsep{1em}
                \item Spalten in die 20 verschiedenen Aminosäuren
                \item Aufnahme durch die Darmzotten im Dünndarm
                \item Transport über Blut in die Leber (Pfortader)
            \end{itemize}
        \end{block}
    \end{frame}

    \begin{frame}[allowframebreaks]
        \frametitle{Fette}

        \underline{Kleinster Baustein: Fettsäuren}
        \begin{block}{Fakten}
            \begin{itemize}
                \setlength\itemsep{1em}
                \item Fest bei Raumtemperatur $\rightarrow$ Fett
                \item Flüssig bei Raumtemperatur $\rightarrow$ Öl
                \item Gute Fette blockieren teilweise den Prozess, der aus übrigem Zucker Fett synthetisiert
                \item Energiedichte: 9 kcal/g
            \end{itemize}
        \end{block}

        \framebreak

        \begin{block}{Aufnahme}
            \begin{itemize}
                \setlength\itemsep{1em}
                \item Spalten in Fettsäuren
                \item Aufnahme durch Lymphgefäße
                \item Lymphgefäße laufen zusammen, sammeln Fett und geben dies dann an das Herz weiter
                \begin{itemize}
                    \item Keine vorzeitige Überprüfung auf gesunde und ungesunde Fette!
                    \item Alle Zellen sind schlechten Fetten ausgeliefert, bis dieses zufällig mal an der Leber vorbeikommt
                \end{itemize}
            \end{itemize}
        \end{block}
    \end{frame}

    \section{Praxis}
    {
        \setbeamercolor{normal text}{fg=hBlue}\usebeamercolor*{normal text}
        \begin{frame}
            \begin{center}
                \Huge Praxis
            \end{center}
        \end{frame}
    }

    \subsubsection{Tipps \& Tricks}
    \begin{frame}[allowframebreaks]
        \frametitle{Tipps \& Tricks}

        \begin{itemize}
            \item KAUEN !!!
        \end{itemize}

        \framebreak

        \begin{itemize}
            \setlength\itemsep{1em}
            \item Fette werden bei Hitze chemisch verändert $\rightarrow$ Nur "`Bratfette"' zum Braten nutzen
            \begin{itemize}
                \item Kokosfett, Butterschmalz, Sesamöl, Ghee, etc.
            \end{itemize}
            \item "`Freie Radikale"' binden sich mit Fett $\rightarrow$ Öl immer direkt wieder verschließen
            \begin{itemize}
                \item Gewollter Effekt soll im Körper stattfinden
            \end{itemize}
            \item Die aktuelle Forschung empfielt 20-30\% der Nahrung sollte aus Fett bestehen
            \begin{itemize}
                \item Vegane Ernährung ist nur mit sehr abwechslungsreicher Nahrung möglich $\rightarrow$ Siehe Aminosäurenprofil
            \end{itemize}
        \end{itemize}

        \framebreak

        \begin{itemize}
            \setlength\itemsep{1em}
            \item Medikamente in Form von Zäpfchen sind immer gesünder als andere Formen, da es nicht über die Leber transportiert wird, die schon einiges herausfiltert\\
            $\rightarrow$ Geringere Dosierung\\
            $\rightarrow$ Weniger Belastung für den Körper
            \item Zucker ist der \underline{einzige} Stoff, der ohne viel Aufwand in Fett umgewandelt werden kann!
        \end{itemize}
    \end{frame}

    \section{Fragerunde}
    {
        \setbeamercolor{normal text}{fg=hBlue}\usebeamercolor*{normal text}
        \begin{frame}
            \begin{center}
                \Huge Fragerunde
            \end{center}
        \end{frame}
    }

\end{document}
