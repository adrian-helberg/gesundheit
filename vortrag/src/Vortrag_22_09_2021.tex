% Preamble
\documentclass[xcolor=dvipsnames]{beamer}
\usetheme{madrid}

% Packages
\usepackage[english,ngerman]{babel}
\usepackage[utf8]{inputenc}
\usepackage{amsmath}
\usepackage{graphicx}
\usepackage{ifthen} % Boolean variables
\usepackage{subfigure} % Horizontal pictures

\definecolor{hBlue}{RGB}{55,118,165}
\usecolortheme[named=hBlue]{structure}

% Distiction between work and stream presentation
\newboolean{work}
\setboolean{work}{true}

\ifthenelse{\boolean{work}}{
    \titlegraphic{\includegraphics[width=4cm]{../images/logo.png}}
}{}
\title{Gesundheit \& Ernährung}
\subtitle{Zusammenfassung}
\ifthenelse{\boolean{work}}{
    \author{Adrian Helberg}
}{
    \author{Bl1nzlar, twitch.tv/bl1nzlar}
}
\date{22.09.2021}


% Document
\begin{document}

    \maketitle

    \frame{\frametitle{Agenda}\tableofcontents}

    \section{Zusammenfassung}
    {
    \setbeamercolor{normal text}{fg=hBlue}\usebeamercolor*{normal text}
    \begin{frame}
        \begin{center}
            \Huge Zusammenfassung
        \end{center}
    \end{frame}
    }

    \subsection{Einführung}
    \begin{frame}
        \frametitle{Einführung}
            \begin{itemize}
                \setlength\itemsep{1em}
                \item Forschung, Fachwissen und Gesellschaft hinterfragen
                \begin{itemize}
                    \item Gesundheit in die eigenen Hände nehmen
                \end{itemize}
                \item Lobbyismus erkennen
                \item Das perfekte Frühstück
                \begin{itemize}
                    \item Das perfekte Spätstück 2.0 (Autophagie, Pfeffer)
                \end{itemize}
                \item Körperlicher und mentaler Reset
            \end{itemize}
    \end{frame}

    \subsection{21-Tage-Kur}
    \begin{frame}
        \frametitle{21-Tage-Kur}
        \begin{itemize}
            \setlength\itemsep{1em}
            \item Teufelskreis durchbrechen
            \begin{itemize}
                \item Zuckerentwöhnung
            \end{itemize}
            \item Hormone ins Gleichgewicht bringen
            \item Organsysteme entlasten
            \item Lebensqualität erlangen
            \begin{itemize}
                \item Chronische Krankheiten heilen
                \item Abnehmen
            \end{itemize}
        \end{itemize}
    \end{frame}

    \subsection{Vitamin D}
    \begin{frame}
        \frametitle{Vitamin D}
        \begin{itemize}
            \setlength\itemsep{1em}
            \item Mangel ausgleichen
            \item Erkrankungen verhindern
            \item Supplementieren
        \end{itemize}
    \end{frame}

    \subsection{Verdauungsrohr}
    \begin{frame}
        \frametitle{Verdauungsrohr}
        \begin{itemize}
            \setlength\itemsep{1em}
            \item Zusammenhang zwischen Verdauung und Gesundheit
            \item Darmbiom
        \end{itemize}
    \end{frame}

    \subsection{Kohenhydrate}
    \begin{frame}
        \frametitle{Kohenhydrate}
        \begin{itemize}
            \setlength\itemsep{1em}
            \item Frucktzucker vermeiden
            \item Blutzuckerspiegel normalisieren
            \item Insulin- und Leptinresistenz
            \begin{itemize}
                \item Stress
                \item Überernährung
                \item Bewegungsmangel
                \item Fettleber
            \end{itemize}
            \item Blutdruck verbessern
            \item Lebensmittel statt Nahrungsmittel
        \end{itemize}
    \end{frame}

    \subsection{Fette}
    \begin{frame}
        \frametitle{Fette}
        \begin{itemize}
            \setlength\itemsep{1em}
            \item Omega-Quotient
            \item Entzündungen loswerden/vermeiden
            \item Potential erkennen
            \item Erkennen von schlechten und guten Fetten
            \item Cholesterin
            \item Fettsäurenprofil
        \end{itemize}
    \end{frame}

    \subsection{Proteine}
    \begin{frame}
        \frametitle{Proteine}
        \begin{itemize}
            \setlength\itemsep{1em}
            \item Gluconeogenese
            \item Biologische Wertigkeit
            \item Muskelaufbau
            \begin{itemize}
                \item Rückenschmerzen lindern/vermeiden
                \item Verletzungsgefahr reduzieren
            \end{itemize}
            \item Aminosäurenprofil
            \item Gluten
            \item mTOR
        \end{itemize}
    \end{frame}

    \subsection{Mayr-Medizin}
    \begin{frame}
        \frametitle{Mayr-Medizin}
        \begin{itemize}
            \setlength\itemsep{1em}
            \item Essensverhalten
            \begin{itemize}
                \item zu schnell, zu viel, zu oft, zu spät, zu schwer
            \end{itemize}
            \begin{itemize}
                \item Das \textit{Wie} ist mindestens genauso wichtig wie das \textit{Was}
            \end{itemize}
            \item Kauen
            \item Bauchmassage
            \item Bauchform und Körperhaltung
            \item Intestinale Autointoxikation
        \end{itemize}
    \end{frame}

    \section{Fragerunde}
    {
        \setbeamercolor{normal text}{fg=hBlue}\usebeamercolor*{normal text}
        \begin{frame}
            \begin{center}
                \Huge Fragerunde
            \end{center}
        \end{frame}
    }
\end{document}