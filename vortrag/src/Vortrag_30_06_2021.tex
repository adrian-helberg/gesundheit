% Preamble
\documentclass[xcolor=dvipsnames]{beamer}
\usetheme{madrid}

% Packages
\usepackage[english,ngerman]{babel}
\usepackage[utf8]{inputenc}
\usepackage{amsmath}
\usepackage{graphicx}
\usepackage{ifthen} % Boolean variables
\usepackage{subfigure} % Horizontal pictures

\definecolor{hBlue}{RGB}{55,118,165}
\usecolortheme[named=hBlue]{structure}

% Distiction between work and stream presentation
\newboolean{work}
\setboolean{work}{true}

\ifthenelse{\boolean{work}}{
    \titlegraphic{\includegraphics[width=4cm]{../images/logo.png}}
}{}
\title{Gesundheit \& Ernährung}
\subtitle{Makronährstoffe II: Fette}
\ifthenelse{\boolean{work}}{
    \author{Adrian Helberg}
}{
    \author{Bl1nzlar, twitch.tv/bl1nzlar}
}
\date{30.06.2021}


% Document
\begin{document}

    \maketitle

    \frame{\frametitle{Agenda}\tableofcontents}

    \section{Theorie}
    {
    \setbeamercolor{normal text}{fg=hBlue}\usebeamercolor*{normal text}
    \begin{frame}
        \begin{center}
            \Huge Theorie
        \end{center}
    \end{frame}
    }

    \subsection{Profil}
    \begin{frame}[allowframebreaks]
        \frametitle{Fett im Profil}

        \begin{figure}
            \centering
            \includegraphics[width=10cm]{../images/fett.png}
            \caption{Schlagworte in Verbindung mit Fett}
        \end{figure}

        \framebreak

        \begin{block}{Was ist Fett?}
            \begin{itemize}
                \setlength\itemsep{1em}
                \item Fettsäuren als kleinste Bausteine von Fett
                \item Die meisten Fettsäuren können (wie Traubenzucker) in den meisten Zellen des Körpers zu Energie verbrannt werden
                \item Einige Fettsäuren können in großen Mengen im Fettgewebe gespeichert werden
                \item Manche Fettsäuren werden als Baustoff für bspw. Zellwende benötigt
            \end{itemize}
        \end{block}

        \begin{figure}
            \centering
            \includegraphics[width=2cm]{../images/fs.png}
            \caption{Fettsäure}
        \end{figure}

        \framebreak

        \begin{figure}
            \centering
            \includegraphics[width=10cm]{../images/fs.jpg}
            \caption{Aufbau von Fettsäuren}
        \end{figure}

        \framebreak

        \begin{figure}
            \centering
            \includegraphics[width=10cm]{../images/fs_2.png}
            \caption{Fettsäuren mit unterschiedlich vielen Kohlenstoffatomen}
        \end{figure}
    \end{frame}

    \subsection{Sättigung}
    \begin{frame}[allowframebreaks]
        \frametitle{Sättigung von Fettsäuren}

        \begin{figure}
            \centering
            \includegraphics[width=8cm]{../images/fs_3.jpeg}
            \includegraphics[width=8cm]{../images/fs_4.png}
            \caption{Gesättigte (oben) und ungesättigte (unten) Fettsäure}
        \end{figure}

        \framebreak

        \begin{block}{Gesättigte Fettsäuren \ldots}
            \begin{itemize}
                \setlength\itemsep{1em}
                \item sind bei Raumtemperatur fest (bspw. Butter)
                \item können zu Energie verbrannt werden
                \item können im Fettgewebe gespeichert werden
                \item können selbst hergestellt werden
                \item können den Cholesterienspiegel erhöhen
            \end{itemize}
        \end{block}

        \framebreak

        \begin{block}{Ungesättigte Fettsäuren \ldots}
            \begin{itemize}
                \setlength\itemsep{1em}
                \item sind bei Raumtemperatur flüssig (bspw. Olivenöl)
                \item werden in einfache und mehrfache Un-sättigung eingeteilt
            \end{itemize}
        \end{block}

        \begin{block}{Mehrfach ungesättigte Fettsäuren \ldots}
            \begin{itemize}
                \setlength\itemsep{1em}
                \item werden nicht zu Energie verbrannt
                \item sind Baustoff für Zellwände und Immunstoffe
                \item können nicht vom Körper selbst hergestellt werden
                \item sind lebenswichtige Nährstoffe
                \item werden unterschieden in Omega-3 und Omega-6
            \end{itemize}
        \end{block}

        \framebreak

        \begin{figure}
            \centering
            \includegraphics[width=10cm]{../images/nahrungsfette.png}
            \caption{Kategorisierung Nahrungsfette}
        \end{figure}
    \end{frame}

    \subsection{Nahrungsfette}
    \begin{frame}[allowframebreaks]
        \frametitle{Nahrungsfette}

        \begin{figure}
            \centering
            \includegraphics[width=10cm]{../images/fettprofile.png}
            \caption{Fettprofile einiger Lebensmittel}
        \end{figure}

    \end{frame}

    \subsection{Eicosanoide}
    \begin{frame}[allowframebreaks]
        \frametitle{Eicosanoide}

        \begin{block}{Eicosanoide \ldots}
            \begin{itemize}
                \setlength\itemsep{1em}
                \item werden für die Regulierung von Entzündungen gebraucht
                \item sind Botenstoffe des Immunsystems
            \end{itemize}
        \end{block}

        \begin{figure}
            \centering
            \includegraphics[width=8cm]{../images/eico.png}
            \caption{Verschiedene Eicosanoide}
        \end{figure}

        \framebreak

        \begin{figure}
            \centering
            \includegraphics[width=10cm]{../images/eico_1.jpg}
            \caption{Konkurrierender Umbau von Omega-Fettsäuren zu Eicosanoiden}
        \end{figure}

        \framebreak

        \begin{figure}
            \centering
            \includegraphics[width=10cm]{../images/eico_2.png}
            \caption{Fetter Seefisch ist besonders wertvoll}
        \end{figure}

    \end{frame}

    \subsection{Nachteile von MUFS}
    \begin{frame}
        \frametitle{Nachteile von MUFS}

        \begin{block}{Mehrfach ungesättigte Fettsäuren \ldots}
            \begin{itemize}
                \setlength\itemsep{1em}
                \item sind sehr reaktionsfreudig
                \item verderben schnell
                \item sind anfällig für Oxidationsprozesse
                \item müssen duch reichlich Antioxidantien geschützt werden
                \item können bei umbedachter Supplementation fegährlich werden (Fischöl-Kapseln)
            \end{itemize}
        \end{block}
    \end{frame}

    \subsection{Tagesbedarf von MUFS}
    \begin{frame}
        \frametitle{Tagesbedarf von MUFS}

        \begin{figure}
            \centering
            \includegraphics[width=4cm]{../images/hering.jpg}
            \includegraphics[width=4cm]{../images/leinsamen.jpg}
            \includegraphics[width=4cm]{../images/walnüsse.jpg}
            \caption{40g Hering ODER 10g Leinsamen ODER 20g Walnüsse}
        \end{figure}
    \end{frame}

    \subsection{Cholesterin}
    \begin{frame}[allowframebreaks]
        \frametitle{Cholesterin}

        \begin{block}{Vorteile: Cholesterin \ldots}
            \begin{itemize}
                \setlength\itemsep{1em}
                \item hat fettähnliche Eigenschaften
                \item wird vom Körper selbst hergestellt
                \item baut Zellwände auf
                \item produziert Gallensäure
                \item bildet Hormono
                \item erfüllt also lebenswichtige Aufgaben
            \end{itemize}
        \end{block}

        \framebreak

        \begin{block}{Nachteile: Cholesterin \ldots}
            \begin{itemize}
                \setlength\itemsep{1em}
                \item kann in Wände von Blutgefüßen eindringen, was zur Erkrankung Arteriosklerose führt
            \end{itemize}
        \end{block}

        \begin{block}{Arteriosklerose \ldots}
            \begin{itemize}
                \setlength\itemsep{1em}
                \item führt zu Herzinfakt
                \item ist die häufigste Todesursache der westlichen Welt
                \begin{itemize}
                    \item jeder zweite Mensch über 65 verstirbt
                    \item jeder dritte Mensch unter 65 verstirbt
                \end{itemize}
            \end{itemize}
        \end{block}

    \end{frame}

    \section{Praxis}
    {
        \setbeamercolor{normal text}{fg=hBlue}\usebeamercolor*{normal text}
        \begin{frame}
            \begin{center}
                \Huge Praxis
            \end{center}
        \end{frame}
    }

    \begin{frame}{Tipps \& Tricks}
        \begin{itemize}
            \setlength\itemsep{1em}
            \item Omega-6 reiche Nahrungsmittel redizieren
            \begin{itemize}
                \item Reaffinierte Pflanzenöle, wie Sonnenblumenöl, Maiskeimöl, etc.
                \item Magarine
                \item Fertigprodukte (vor allem Backwaren)
            \end{itemize}
            \item Omega-3 reiche Nahrungsmittel gezielt in die Ernährungs einbauen
            \begin{itemize}
                \item Fetter Seefisch, wie Lachs, Makrele, etc.
                \item Leinsamen
                \item Nüsse, wie Walnüsse, Haselnüsse, etc.
            \end{itemize}
        \end{itemize}
    \end{frame}

    \section{Fragerunde}
    {
        \setbeamercolor{normal text}{fg=hBlue}\usebeamercolor*{normal text}
        \begin{frame}
            \begin{center}
                \Huge Fragerunde
            \end{center}
        \end{frame}
    }

\end{document}